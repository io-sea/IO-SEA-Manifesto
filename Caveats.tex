\chapter{Caveats and things to be completed}

This chapter describes the caveats encountered during the run of the project and how they were solved and/or
worked around.

\paragraph{}
Other aspects appeared during the run of the project: some collateral features were not foreseen at the time the
proposal was written, which means that no "PM time" was allocated to developing those tasks. They were not 
implemented in IO-SEA but "placeholders" have been installed to implement their future implementation.  This could
be part of a follow-up to IO-SEA, leveraging the outcomes of IO-SEA. 

\section{Disappearance of MOTR}

The MOTR object stored was abandoned by Seagate during the run of the project, and it was no longer maintained as
an open-source software. The IO-SEA project did not have the required task force to perform that maintenance, so 
it was decided to stop using MOTR.

This action was not very impacting. The interfaces between the different components were based, as much as 
possible, on well established standards. In particular, it was decided to avoid to use API only used in a single
product when alternatives existed, or use the specific API as well as a more "opened" one. This is why all the
interactions with MOTR could be made via the dedicated MOTR API as well as S3. MOTR came with an efficient S3 
server which helped in following that path. 

When the specific MOTR API had to used, this was isolated in a well identified layer. This was done in the 
KVSNS library and inside the DASI. 

Finally, replacing MOTR by some other object stores was not a problem. It has been proofed that other object
store, such as CEPH/RADOS could be used instead of MOTR. Other object stores, like DAOS, could be used as well 
with relatively minor interfaces to be implemented. The "coding surface" is rather small and clearly stated in
the different WP's deliverables. 

\section{Authentication and user management}

IO-SEA does not explicitly manage the users. It embeds no specific user authentication, nor explicit API to 
define precisely who can access the object, not define Access Control Lists (ACLs). Io-SEA relies on the user
management in each of this component. This was no blocker in the run of the project, for every component was
based on similar and/or compatible paradigms, but a more rigorous approach should implement such a feature. 

This aspect was not foreseen as the proposal was written, it became visible as the project advanced. Since the 
situation did not lead to issues, it was decided to implement to authentication and user management, but to 
implement, inside the IO-SEA software stack, the required "placeholders".


Placeholders are empty functions, or commented calls in the code, describing where authentication and user 
management should be considered. A later implementation of those features will benefit from those placeholders, 
saving the time to identify the point of contacts with the existing software.

Implementation of Authentication and User Management should be done by creating at least two new components:
\begin{itemize}
    \item the \textbf{Key Master}:~this component should manage authentication. A credential structure should be
    defined, and this component should provide ways to create this structure, via information provided by the 
    user. This could be typically be done by leveraging paradigms from the \textit{Kerberos~5} ecosystem.
    
    \item the \textbf{Gate Keeper}:~each dataset, namespace, or even objects in IO-SEA, should be associated
    with an Access Control List (ACL) that describes who can access it in which manner. For example, one could
    allow all the members of a group to read it, not allow updates or modifications only to a few users (seen 
    as "administrators of this entity")). The users wanting to access an entity should expose their credentials
    to have it compared to the entity's ACL. If this matches, the user is allowed and can perform the action, 
    otherwise the access is denied and the user receives an error. 

\end{itemize}

Since the architecture of the IO-SEA is designed to run on exascale ready system, there should be no bottleneck
which means that Gate Keeper and Key Master should have multiple instances, in order to keep the storage's
scalability on both data and metadata. This is not a simple challenge, for security feature generally enforce the
use of a central, not replicated, database describing the given authorizations. 

\section{Accounting and storage space management}

This feature is strongly related to the previous one. As user are clearly identified by their credentials, it
becomes easy, from a central point of view, to keep tracks of the storage volume, form both a data and a metadata
perspective, are allocated to or consumed by a user, a group of users or a project. As this storage space is 
accounted, it becomes possible to implemented limits and fences, making it possible to avoid "over burning" of
resources by some users, leading to starve other users. Too greedy users should be blocked, via a dedicated 
error, similar to POSIX error ENOSPC, blocking the user accesses. 

The instant use of network resources (typically the network bandwidth) can be accounted as well, aside the 
storage media use. 

Implementing such a feature is quite a major challenge, like accounting it requires a central database which
is difficult to replicate. Having a strict space management in a massively distributed architecture as the
IO-SEA one lead to apparently contradictory choices. This would require to study which smart compromises could
be adopted in order to define a new component to be implemented. 

\section{Expand the concept of dataset and namespace}

The workflow management is based on two concepts: the \textbf{datasets} and the \textbf{namespaces}. 

Datasets are "baskets full of data", they gather pieces of data that have "good reasons" to be used together. For
example, a dataset could represent the results of an experiment, or the input or results of a simulation run. 

Datasets are currently unstructured. A specific object exists inside one and only one dataset, it is never shared
between two different dataset. The dataset drive the accesses to data, compute jobs need some datasets to be 
used in a read-only or read-write manner and the workflow manager makes sure that no conflict occurs during that 
action. 

Namespaces are implementing given semantics on top of datasets. For example, a POSIX namespace will provide a 
POSIX interface to the dataset, and a S3 namespace will bring an S3 interface. Fundamentally, one can consider
that the difference between datasets and namespaces is similar to the difference between data and metadata: 
namespaces describe how to organise metadata in order to access the data belonging to a dataset. Several 
namespaces can theoretically be associated with a dataset. 

An ephemeral service implements and instantiates namespaces on top of datasets. For example, the NFS-Ganesha
ephemeral service will implement a NFS namespace on top of datasets and bring NFS semantics to client compute 
nodes. 

\subsection{Making namespaces aware of the Key Master and the Gate Keeper}

The namespaces often maps objects inside object stores to another semantics. For example, POSIX or NFS could map 
objects as files. Records in the namespaces have ACLs, that should be compliant with the ACLs of the dataset as
managed by the Gate Keeper. For example, if a dataset is access in read-only and exposed via NFS, then all files
should have only read-only compatible ACLs. 

\subsection{Subsets inside datasets}

Current implementation of datasets only contains objects. So far, they are very similar to S3 buckets. It would
be natural to make datasets capable of containing other datasets. This approach makes datasets closer to POSIX
directories. This introduces the ideas that datasets have subsets, which are themselves datasets, dataset 
definition would become recursive.

In such a situation, a "sub dataset" could have its own ACLs. How should it be considered if a different and
potentially conflicting ACL is exposed in a "lower" or "upper" dataset. Policies are to be defined, such as 
ways to define such policies. Once a choice is made, impacts on the implementation of the Gate Keeper and the 
Key Master should be defined and managed. 

