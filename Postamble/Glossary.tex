%%%%%%%%%%%%%%%%%%%%%%%%%%%%%%%%%%%%%%%%%%%%%%
%%%%%%%% place glossary entries below %%%%%%%%%


%%%% NOTE!!!!!!!!!!!!

 %%% I think you MUST mark a glossary term with \Gls{} in the text for it to appear in the glossary


\glssetwidest[0]{blahblahblahblah}

\newglossaryentry{Hercule}{
    name=Hercule,
    description={Parallel I/O and data management library developed at CEA}
}

\newglossaryentry{AMR}{
    name=AMR,
    description={Adaptive Mesh Refinement} 
}

\newglossaryentry{RAMSES}{
    name=RAMSES,
    description={Code to model astrophysical systems, featuring self-gravitating, magnetised, compressible, radiative fluid flow, using AMR technique.}
}

\newglossaryentry{JSC}{
    name=JSC,
    description={The J\"{u}lich Supercomputing Centre at the Forschungszentrum  J\"{u}lich is one of the IO-SEA partners}
}
\newglossaryentry{FZJ}{
    name=FZJ,
    description={Forschungszentrum J\"{u}lich, in J\"{u}lich, Germany, is one of the largest research centres in Europe and a member of the Helmholtz Association}
}

\newglossaryentry{LQCD}
{
    name=LQCD,
    description={Lattice quantum-chromodynamics is a numerical framework for calculating physical properties of hadrons, composite particles composed of quarks}
}

\newglossaryentry{TSMP}
{
    name=TSMP,
    description={Terrestrial System Modelling Platform is an open source scale-consistent, highly modular, massively parallel regional Earth system model}
}

\newglossaryentry{COSMO}
{
    name=COSMO,
    description={Consortium for Small-scale Modeling}
}

\newglossaryentry{CLM}
{
    name=CLM,
    description={Community Land Model}
}

\newglossaryentry{CLI}
{
    name=CLI,
    description={Command Line Interface}
}

\newglossaryentry{ParFlow}
{
    name=ParFlow,
    description={A physically-based and spatially distributed hydrological model solving surface and subsurface flows in a massively parallel computational framework}
}
\newglossaryentry{OASIS3-MCT}
{
    name=OASIS,
    description={OASIS3-MCT is a software allowing synchronized exchanges of coupling information between numerical codes representing different components of the Earth System}
}
\newglossaryentry{MPMD}
{
    name=MPMD,
    description={Multiple Program Multiple Data}
}

\newglossaryentry{IFS}
{
    name=IFS,
    description={Integrated Forecasting System, ECMWF's operational weather forecasting system}
}
\newglossaryentry{MARS}
{
    name=MARS,
    description={The Meteorological Archival and Retrieval System is ECMWF's perpetual archive service}
}
\newglossaryentry{RAM}
{
    name=RAM,
    description={Random Access Memory  is memory optimised for random access, often by a compute device}
}

\newglossaryentry{NetCDF}
{
    name=NetCDF,
    description={NETwork Common Data Form is a community standard, machine-independent data format that support the creation, access, and sharing of array-oriented scientific data. It is extensively used in Earth system modelling}
}
\newglossaryentry{NVRAM}
{
    name=NVRAM,
    description={Non-volatile \Gls{RAM} is \Gls{RAM} that does not lose the stored information after a short time without constant refreshing}
}

\newglossaryentry{MSA}
{
    name=MSA,
    description={Modular Supercomputing Architecture}
}
\newglossaryentry{DA}
{
    name=DA,
    description={Data Assimilation}
}
\newglossaryentry{CUDA}{
    type=\acronymtype,
    name=CUDA,
    description={The Compute Unified Device Architecture is a parallel computing platform as well as an API that allows for the communication with certain types of graphics-processing units}
}
\newglossaryentry{CPU}
{
    name=CPU,
    description={Central Processing Unit}
}
\newglossaryentry{GPU}
{
    name=GPU,
    description={Graphics Processing Unit}
}
\newglossaryentry{SDLTS}
{
    name=SDLTS,
    description={Simulation and Data Laboratory: Terrestrial Systems}
}
\newglossaryentry{ECMWF}
{
    name=ECMWF,
    description={European Centre for Medium-Range Weather Forecasts}
}
\newglossaryentry{ParTec}
{
    name=ParTec,
    description={ParTec is one of the leading SMEs in the HPC domain in Europe}
}
\newglossaryentry{IT4I}
{
    name=IT4I,
    description={IT4Innovations National Supercomputing Centre at VSB Technical University of Ostrava, Czech Republic}
}
\newglossaryentry{ATOS}
{
    name=ATOS,
    description={ATOS is Europe's largest digital services deliverer}
}
\newglossaryentry{Eviden}
{
    name=Eviden,
    description={Eviden, formerly ATOS, is Europe's largest digital services deliverer}
}
\newglossaryentry{CEA}
{
    name=CEA,
    description={The French Alternative Energies and Atomic Energy Commission}
}
\newglossaryentry{PDAF}
{
    name=PDAF,
    description={Parallel Data Assimilation Framework}
}

\newglossaryentry{CEITEC}
{
    name=CEITEC,
    description={Central European Institute of Technology, Masaryk University, Brno, Czech Republic}
}


\newglossaryentry{MU}
{
    name=MU,
    description={Masaryk University, Brno, Czech Republic}
}


\newglossaryentry{iRODS}
{
    name=iRODS,
    description={Integrated Rule-Oriented Data System}
}

\newglossaryentry{PID}
{
    name=PID,
    description={Persistent Identifier}
}
\newglossaryentry{HSM}
{
    name=HSM,
    description={Hierarchical Storage Management}
}
\newglossaryentry{S3}
{
    name=S3,
    description={Amazon's Simple Storage Service: HTTP-based protocol to access data. Initially developed by Amazon, it generalisation made it a de facto standard for data access in cloud services}
}
\newglossaryentry{TCO}
{
    name=TCO,
    description={Total Cost of Ownership}
}
\newglossaryentry{SSD}
{
    name=SSD,
    description={Solid State Drive}
}
\newglossaryentry{HDD}
{
    name=HDD,
    description={Hard Drive Disk}
}
\newglossaryentry{HTTP}
{
    name=HTTP,
    description={HyperText Transfer Protocol: common Internet protocol to access web pages or download files}
}
\newglossaryentry{NVMe}
{
    name=NVMe,
    description={Non-Volatile Memory Express}
}
\newglossaryentry{NVMe-oF}
{
    name=NVMe-oF,
    description={NVM Express over Fabrics}
}
\newglossaryentry{Swift}
{
    name=Swift,
    description={Object-based interface of the OpenStack suite}
}
\newglossaryentry{POSIX}
{
    name=POSIX,
    description={Portable Operating System Interface is a family of standards specified by the IEEE Computer Society for maintaining compatibility between operating systems}
}
\newglossaryentry{API}{
    type=\acronymtype,
    name=API,
    description={An Application Programming Interfaces (API) allows software to communicate with other software which support the same API}
}
\newglossaryentry{DASI}{
    name=DASI,
    description={Data Access and Storage Interface developed in Work Package 5}
}

\newglossaryentry{SLURM}{
    name=SLURM,
    description={SLURM is an open-source cluster-management and job-scheduling system}
}

\newglossaryentry{JUBE}{
    type=\acronymtype,
    name=JUBE,
    description={The J\"{U}lich Benchmarking Environment is a script based framework to easily create benchmark sets, run those sets on different computer systems and to evaluate the results}
}

\newglossaryentry{CI/CD}{
    name=CI/CD,
    description={Continuous Integration/Continuous Deployment is an automated system for the testing, integration and deployment of software}
}

\newglossaryentry{CM}{
    name=CM,
    description={Cluster Module - the general purpose compute module of the DEEP System}
}

\newglossaryentry{ESB}{
    name=ESB,
    description={Extreme Scale Booster - module of the DEEP System focused on compute intensive applications with high scalability}
}

\newglossaryentry{DAM}{
    name=DAM,
    description={Data Analytics Module of the DEEP System focused on data-intensive applications}
}

\newglossaryentry{SSSM}{
    name=SSSM,
    description={Scalable Storage Service Module - conventional, spinning-disk-based storage module of the DEEP System}
}

\newglossaryentry{AFSM}{
    name=AFSM,
    description={The All-Flash Storage Module is a purely flash-based storage module of the \Gls{DEEP} system}
}

\newglossaryentry{Kronos}{
    name=Kronos,
    description={The ECMWF workload simulator}
}

\newglossaryentry{BDGS}{
    type=\acronymtype,
    name=BDGS,
    description={The Big Data Generator Suite efficiently generates scalable big data while employing data models derived from real data to preserve data veracity \cite{ming2013bdgs}}
}

\newglossaryentry{FDB}{
    name=FDB,
    description={Fields DataBase, the ECMWF meteorological object store}
}

\newglossaryentry{MPI}{
    type=\acronymtype,
    name=MPI,
    description={The Message Passing Interface is a common API for communication between tasks running on one or more computers}
}

\newglossaryentry{I/O}{
    type=\acronymtype,
    name=I/O,
    description={Input/Output is either a noun referring to the action of doing either input and/or output, generally either reading or writing memory, or is an adjective or adverb that describes that the following operation does input and/or output}
}

\newglossaryentry{DEEP}{
    type=\acronymtype,
    name=DEEP,
    description={The Dynamical Exascale Entry Platform (DEEP) is a multi-component project project to prepare for upcoming exascale HPC systems. The DEEP-SEA project is latest component of this project. It is also used to refer to the prototype developed in the DEEP projects based on context.}
}

\newglossaryentry{Git}{
    type=\acronymtype,
    name=Git,
    description={Git is a software for tracking changes to files that is often used amon programmers to coordinate work}
}
\newglossaryentry{GitLab}{
    type=\acronymtype,
    name=GitLab,
    description={GitLab is an open-source software-development--and--operations platform that uses Git to organize software development}
}

\newglossaryentry{OSU}{
    type=\acronymtype,
    name=OSU,
    description={The Ohio State University is a university in the state of Ohio in the United States of America}
}
\newglossaryentry{TCP}{
    type=\acronymtype,
    name=TCP,
    description={The Transmission Control Protocol  is one of the main communication protocols of the internet protocol}
}
\newglossaryentry{IB verbs}{
    type=\acronymtype,
    name=IB verbs,
    description={The API for communication using InfiniBand, a communication hardware}
}
\newglossaryentry{UCP}{
    type=\acronymtype,
    name=UCP,
    description={The Unified Communication Protocol is an API aimed at unifying different communication APIs, similiar in that sense to MPI}
}
\newglossaryentry{PSM2}{
    type=\acronymtype,
    name=PSM2,
    description={Performance Scaled Messaging 2 is the second generation of the PSM API for communication}
}
\newglossaryentry{NVLink}{
    type=\acronymtype,
    name=NVLink,
    description={NVLink describes the suite of tools for communicating between NVIDIA GPUs. It includes an API that requires physical NVLink bridges between GPUs to use}
}
\newglossaryentry{OMB}{
    type=\acronymtype,
    name=OMB,
    description={The Ohio State University (OSU) MicroBenchmarks is a suite of \Gls{MPI} benchmarks that has been extended to other communication \Glspl{API}}
}

\newglossaryentry{QIO}{
    type=\acronymtype,
    name=QIO,
    description={QCD Input/Output Applications Programmer Interface developed under the auspices of the U.S. Department of Energy Scientific Discovery through Advanced Computing (SciDAC) program}
}

\newglossaryentry{CHROMA}{
    type=\acronymtype,
    name=CHROMA,
    description={The Chroma software system for lattice QCD }
}

\newglossaryentry{ICHEC}{
    type=\acronymtype,
    name=ICHEC,
    description={Irish Centre for High-End Computing is the national HPC centre in Ireland.}
}
\newglossaryentry{FFT}{
    type=\acronymtype,
    name=FFT,
    description={The Fast Fourier Transform was originally an optimized algorithm proposed for Fourier transforming discrete data, nowadays it more commonly refers to a suite of better-optimized algorithms that perform said transform}
}

\newglossaryentry{SQL}{
    type=\acronymtype,
    name=SQL,
    description={The Structured Query Language  is a domain-specific language for accessing data in a \Gls{RDBMS} or in a \Gls{RDSMS}}
}

\newglossaryentry{RDBMS}{
    type=\acronymtype,
    name=RDBMS,
    description={A Relational-DataBase Management System (RDBMS) is a management system for relational databases}
}

\newglossaryentry{RDSMS}{
    type=\acronymtype,
    name=RDBMS,
    description={A Relational-Data--Stream Management System (RDSMS) is a management system for relational data streams}
}

\newglossaryentry{NoSQL}{
    type=\acronymtype,
    name=NoSQL,
    description={NOn relational \Gls{SQL}  databases are SQL databases that can efficiently handle huge amounts of unstructured rapidly changing data. NoSQL unlike SQL does not refer to a language and is generally an adjective}
}

\newglossaryentry{RED-SEA}{
    type=\acronymtype,
    name=RED-SEA,
    description={EuroHPC project focused on network solutions for exascale architectures}
}

\newglossaryentry{PCIe}{
    type=\acronymtype,
    name=PCIe,
    description={Peripheral Component Interconnect Express is a high-speed serial-bus expansion standard designed to unify and replace older standards}
}

\newglossaryentry{JURECA-DC}{
    type=\acronymtype,
    name=JURECA-DC,
    description={The J\"{U}lich Research on Exascale Cluster Architectures -- Date Centre  module is one of the modules of the J\"{U}lich Research on Exascale Cluster Architectures (JURECA) system}
}

\newglossaryentry{JUWELS}{
    type=\acronymtype,
    name=JUWELS,
    description={The J\"{U}lich Wizard for European Leadership Science is one of the super computers at \Gls{JSC}}
}

\newglossaryentry{HPL}{
    type=\acronymtype,
    name=HPL,
    description={The High Performance \Gls{LINPACK} is a performant software package for solving linear system}
}

\newglossaryentry{LINPACK}{
    type=\acronymtype,
    name=LINPACK,
    description={LINPACK is a software package for solving linear systems}
}

\newglossaryentry{FLOPS}{
    type=\acronymtype,
    name=FLOPS,
    description={FLoating-point OPerations per Second (FLOPS) is the number of floating-point operations achieved in a second}
}

\newglossaryentry{SEA}{
    type=\acronymtype,
    name=SEA,
    description={The Software/Solutions for Exascale Architectures  project is a joint combination of three separate projects \Gls{DEEP}-, IO-, and RED-SEA}
}

\newglossaryentry{HBM2}{
    type=\acronymtype,
    name=HBM2,
    description={The second generation of High Bandwidth Memory  using synchronous dynamic random-access memory stacked in three dimensional space}
}

\newglossaryentry{TIFF}{
    type=\acronymtype,
    name=TIFF,
    description={The Tag Image File Format  is an image file format}
}

\newglossaryentry{HPC}{
    type=\acronymtype,
    name=HPC,
    description={High-Performance Computing}
}

\newglossaryentry{AI}{
    type=\acronymtype,
    name=AI,
    description={Artificial Intelligence}
}

\newglossaryentry{IOR}{
    type=\acronymtype,
    name=IOR,
    description={The Interleaved Or Random (IOR) benchmark is benchmark for \Gls{I/O}}
}

\newglossaryentry{MPI-I/O}{
    type=\acronymtype,
    name=MPI-I/O,
    description={\Gls{MPI} -- Input/Output is an extentsion to \Gls{MPI} for \Gls{I/O}}
}

\newglossaryentry{GPFS}{
    type=\acronymtype,
    name=GPFS,
    description={The General Parallel File System (GPFS) is an IBM developed high-performance clustered file system}
}

\newglossaryentry{Lustre}{
    type=\acronymtype,
    name=Lustre,
    description={Lustre is an open-source parallel file system}
}

\newglossaryentry{mdtest}{
    type=\acronymtype,
    name=mdtest,
    description={Included with the \Gls{IOR} benchmark, the mdtest benchmark is for benchmarking metadata creation}
}

\newglossaryentry{Linktest}{
    type=\acronymtype,
    name=Linktest,
    description={Linktest is a communication-API benchmark developed at \Gls{JSC}}
}

\newglossaryentry{FPGA}{
    type=\acronymtype,
    name=FPGA,
    description={Field-Programmable Gate Array}
}

\newglossaryentry{QCD}{
    type=\acronymtype,
    name=QCD,
    description={Quantum-ChromoDynamics (QCD)}
}

\newglossaryentry{GitLab Runner}{
    type=\acronymtype,
    name={GitLab Runner},
    description={A GitLab Runner is software that connects to GitLab servers for remote execution of CI/CD tasks}
}

\newglossaryentry{STREAM2}{
    type=\acronymtype,
    name={STREAM~2},
    description={A benchmark for measuring \Gls{CPU}-cache and \Gls{CPU}-to-\Gls{RAM} latencies and bandwidth}
}

\newglossaryentry{HPCG}{
    type=\acronymtype,
    name={HPCG},
    description={The High Performance Conjugate Gradient benchmark is a benchmark based on a conjugate-gradient kernel}
}

\newglossaryentry{HMC}{
    type=\acronymtype,
    name={HMC},
    description={The Hamiltonian Monte Carlo algorithm (originally known as hybrid Monte Carlo) is a Markov chain Monte Carlo method for obtaining a sequence of random samples which converge to being distributed according to a target probability distribution for which direct sampling is difficult}
}
\newglossaryentry{WDF}{
    type=\acronymtype,
    name={WDF},
    description={The workflow description file is a YAML configuration file describing ephemeral services and steps invoked in a workflow}
    }

\newglossaryentry{WFM}{
    type=\acronymtype,
    name={WFM},
    description={The IO-SEA Workflow Manager starts ephemeral storage services, and runs workflow steps in the IO-SEA storage environment.
    %\fxnote{ Maybe polish the WFM definition.}
    }
    }
\newglossaryentry{SBB}{
    type=\acronymtype,
    name={SBB},
    description={Smart Burst Buffer is a hardware
accelerator that can be included in the I/O data
path and accelerate I/O on specific files for
targeted applications. Offered as an IO-SEA ephemeral service.
}}

\newglossaryentry{IOI}{
    type=\acronymtype,
    name={IOI},
    description={The IO Instrumentation tool collects I/O-related metrics and provides  diagnostic information about workflow read and wrinte operations}
}
\newglossaryentry{NFS}{
    type=\acronymtype,
    name={NFS},
    description={Network File System, a file system allowing to share files between many nodes over a TCP/IP network}
}
\newglossaryentry{mmap}{
    type=\acronymtype,
    name={mmap},
    description={mmap is a POSIX-compliant Unix system call that maps files or devices into memory}
}


\newglossaryentry{PFB}{
    type=\acronymtype,
    name={PFB},
    description={ParFlow Binary format is a binary file format which is used to store ParFlow grid data}
}


\newglossaryentry{NCL}{
    type=\acronymtype,
    name={NCL},
    description={The NCAR Command Language is an open source interpreted language, designed specifically for scientific data processing and visualization}
}

